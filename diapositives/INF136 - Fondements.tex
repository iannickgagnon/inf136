\documentclass{beamer}

% Pour les citations
\usepackage{csquotes}

% Pour le code
\usepackage{minted}

% Page de garde
\title{INF136 - Fondements}
\author{Iannick Gagnon}
\institute{École de technologie supérieure}
\date{Été 2025}

\begin{document}

\frame{\titlepage}

% SLIDE 1
\begin{frame}[fragile]

    \frametitle{Pourquoi apprendre à programmer?}
    \vspace{1em}
    
    \begin{itemize}
        \item Automatiser ce que l'humain fait mal ou n'aime pas faire.
        \item Résoudre des problèmes trop exigeants ou trop complexes pour le cerveau humain.
        \item Être plus rapide et faire moins d'erreur.
        \newline
    \end{itemize}
    
    \begin{displayquote}
    Les ordinateurs sont rapides, précis et stupides. Les humains sont lents, inexacts et brillants. Ensemble, ils sont puissants au-delà de toute imagination. — Albert Einstein
    \end{displayquote}

    \begin{block}{Remarque}
    Les cours suivants utilisent Python : MEC423, MEC536, MEC744, MEC745 et MEC788.
    \end{block}
    
\end{frame}

% SLIDE 2
\begin{frame}[fragile]

    \frametitle{Terminologie}
    \vspace{1em}
    
    \begin{itemize}
        \item \textbf{Algorithme} : suite d’instructions permettant de résoudre un problème.
        \item \textbf{Pseudo-code} : description en langage naturel d’un algorithme.
        \item \textbf{Programme} : série d’instructions destinées à être exécutées par un ordinateur.
        \item \textbf{Variable} : espace mémoire étiqueté contenant une valeur.
    \end{itemize}
    
\end{frame}

% SLIDE 3
\begin{frame}[fragile]

    \frametitle{Exemples}
    \vspace{1em}
    
    Exemple de pseudo-code:
    \vspace{0.25em}
    \begin{minted}[fontsize=\small]{text}
    FONCTION factorielle(n)
        r ← 1
        POUR i ALLANT DE 2 À n
            r ← r × i
        FIN POUR
        RETOURNER r
    FIN FONCTION
    \end{minted}
    \vspace{0.25em}
    Exemple de programme informatique:
    \vspace{0.25em}
    \begin{minted}[fontsize=\small]{python}
    def factorielle(n):
        r = 1
        for i in range(2, n + 1):
            r *= i
        return r
    \end{minted}

\end{frame}

% SLIDE 4
\begin{frame}[fragile]
    
    \frametitle{Qu’est-ce qu’une variable ?}
    \vspace{0.5em}
    
    \begin{itemize}
      \item Une \textbf{variable} est un nom qui référence un espace mémoire contenant une valeur.
      \item On peut la visualiser comme une \textbf{boîte étiquetée}:
      \begin{enumerate}
        \item La boîte représente l'espace mémoire alloué à la variable.
        \item L'étiquette est l'identifiant ou le nom de la variable.
        \item La valeur correspond au contenu de la boîte.
      \end{enumerate}
      \item Sa valeur peut être changée à tout moment avec \textbf{l'opérateur d'assignation} (\verb|=|).
    \end{itemize}
    
    \vspace{0.25em}

    \begin{block}{Exemple}
    \begin{minted}[fontsize=\small]{python}
    x = 123    # Créer et initialiser la variable x
    x = 456    # Remplacer la valeur contenue dans x
    \end{minted}
    \end{block}

\end{frame}

\end{document}